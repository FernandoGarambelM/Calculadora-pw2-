%package list
\documentclass{article}
\usepackage[top=3cm, bottom=3cm, outer=3cm, inner=3cm]{geometry}
\usepackage{multicol}
\usepackage{graphicx}
\usepackage{url}
%\usepackage{cite}
\usepackage{hyperref}
\usepackage{array}
%\usepackage{multicol}
\newcolumntype{x}[1]{>{\centering\arraybackslash\hspace{0pt}}p{#1}}
\usepackage{natbib}
\usepackage{pdfpages}
\usepackage{multirow}
\usepackage[normalem]{ulem}
\useunder{\uline}{\ul}{}
\usepackage{svg}
\usepackage{xcolor}
\usepackage{listings}
\lstdefinestyle{ascii-tree}{
    literate={├}{|}1 {─}{--}1 {└}{+}1 
  }
\lstset{basicstyle=\ttfamily,
  showstringspaces=false,
  commentstyle=\color{red},
  keywordstyle=\color{blue}
}
%\usepackage{booktabs}
\usepackage{caption}
\usepackage{subcaption}
\usepackage{float}
\usepackage{array}

\newcolumntype{M}[1]{>{\centering\arraybackslash}m{#1}}
\newcolumntype{N}{@{}m{0pt}@{}}


%%%%%%%%%%%%%%%%%%%%%%%%%%%%%%%%%%%%%%%%%%%%%%%%%%%%%%%%%%%%%%%%%%%%%%%%%%%%
%%%%%%%%%%%%%%%%%%%%%%%%%%%%%%%%%%%%%%%%%%%%%%%%%%%%%%%%%%%%%%%%%%%%%%%%%%%%
\newcommand{\itemEmail}{lluquecon@unsa.edu.pe fgarambel@unsa.edu.pe}
\newcommand{\itemStudent}{Luis Guillermo Luque Condori, Fernando Miguel Garambel Marín}
\newcommand{\itemCourse}{Laboratorio de Programación}
\newcommand{\itemCourseCode}{20233478}
\newcommand{\itemSemester}{II}
\newcommand{\itemUniversity}{Universidad Nacional de San Agustín de Arequipa}
\newcommand{\itemFaculty}{Facultad de Ingeniería de Producción y Servicios}
\newcommand{\itemDepartment}{Departamento Académico de Ingeniería de Sistemas e Informática}
\newcommand{\itemSchool}{Escuela Profesional de Ingeniería de Sistemas}
\newcommand{\itemAcademic}{2024 - A}
\newcommand{\itemInput}{Del 4 de mayo 2024}
\newcommand{\itemOutput}{Al 8 de mayo 2024}
\newcommand{\itemPracticeNumber}{01}
\newcommand{\itemTheme}{Git y GitHub}
%%%%%%%%%%%%%%%%%%%%%%%%%%%%%%%%%%%%%%%%%%%%%%%%%%%%%%%%%%%%%%%%%%%%%%%%%%%%
%%%%%%%%%%%%%%%%%%%%%%%%%%%%%%%%%%%%%%%%%%%%%%%%%%%%%%%%%%%%%%%%%%%%%%%%%%%%

\usepackage[english,spanish]{babel}
\usepackage[utf8]{inputenc}
\AtBeginDocument{\selectlanguage{spanish}}
\renewcommand{\figurename}{Figura}
\renewcommand{\refname}{Referencias}
\renewcommand{\tablename}{Tabla} %esto no funciona cuando se usa babel
\AtBeginDocument{%
	\renewcommand\tablename{Tabla}
}

\usepackage{fancyhdr}
\pagestyle{fancy}
\fancyhf{}
\setlength{\headheight}{30pt}
\renewcommand{\headrulewidth}{1pt}
\renewcommand{\footrulewidth}{1pt}
\fancyhead[L]{\raisebox{-0.2\height}{\includegraphics[width=3cm]{img/logo_episunsa.png}}}
\fancyhead[C]{\fontsize{7}{7}\selectfont	\itemUniversity \\ \itemFaculty \\ \itemDepartment \\ \itemSchool \\ \textbf{\itemCourse}}
\fancyhead[R]{\raisebox{-0.2\height}{\includegraphics[width=1.2cm]{img/logo_abet}}}
\fancyfoot[L]{Luis, Fernando, Anthony, William}
\fancyfoot[C]{\itemCourse}
\fancyfoot[R]{Página \thepage}

% para el codigo fuente
\usepackage{listings}
\usepackage{color, colortbl}
\definecolor{dkgreen}{rgb}{0,0.6,0}
\definecolor{gray}{rgb}{0.5,0.5,0.5}
\definecolor{mauve}{rgb}{0.58,0,0.82}
\definecolor{codebackground}{rgb}{0.95, 0.95, 0.92}
\definecolor{tablebackground}{rgb}{0.8, 0, 0}

\lstset{frame=tb,
	language=bash,
	aboveskip=3mm,
	belowskip=3mm,
	showstringspaces=false,
	columns=flexible,
	basicstyle={\small\ttfamily},
	numbers=none,
	numberstyle=\tiny\color{gray},
	keywordstyle=\color{blue},
	commentstyle=\color{dkgreen},
	stringstyle=\color{mauve},
	breaklines=true,
	breakatwhitespace=true,
	tabsize=3,
	backgroundcolor= \color{codebackground},
}

\begin{document}
	
	\vspace*{10px}
	
	\begin{center}	
		\fontsize{17}{17} \textbf{ Informe de Laboratorio \itemPracticeNumber}
	\end{center}
	\centerline{\textbf{\Large Tema: \itemTheme}}
	%\vspace*{0.5cm}	

	\begin{flushright}
		\begin{tabular}{|M{2.5cm}|N|}
			\hline 
			\rowcolor{tablebackground}
			\color{white} \textbf{Nota}  \\
			\hline 
			     \\[30pt]
			\hline 			
		\end{tabular}
	\end{flushright}	

	\begin{table}[H]
		\begin{tabular}{|x{4.7cm}|x{4.8cm}|x{4.8cm}|}
			\hline 
			\rowcolor{tablebackground}
			\color{white} \textbf{Estudiantes} & \color{white}\textbf{Escuela}  & \color{white}\textbf{Asignatura}   \\
			\hline 
			{\itemStudent \par \itemEmail} & \itemSchool & {\itemCourse \par Semestre: \itemSemester \par Código: \itemCourseCode}     \\
			\hline 			
		\end{tabular}
	\end{table}		
	
	\begin{table}[H]
		\begin{tabular}{|x{4.7cm}|x{4.8cm}|x{4.8cm}|}
			\hline 
			\rowcolor{tablebackground}
			\color{white}\textbf{Laboratorio} & \color{white}\textbf{Tema}  & \color{white}\textbf{Duración}   \\
			\hline 
			\itemPracticeNumber & \itemTheme & 04 horas   \\
			\hline 
		\end{tabular}
	\end{table}
	
	\begin{table}[H]
		\begin{tabular}{|x{4.7cm}|x{4.8cm}|x{4.8cm}|}
			\hline 
			\rowcolor{tablebackground}
			\color{white}\textbf{Semestre académico} & \color{white}\textbf{Fecha de inicio}  & \color{white}\textbf{Fecha de entrega}   \\
			\hline 
			\itemAcademic & \itemInput &  \itemOutput  \\
			\hline 
		\end{tabular}
	\end{table}
	
	\section{Actividades}
	\begin{itemize}		
		\item Cree una cuenta de usuario en github
		\item Configure su cuenta de estudiante (https://education.github.com/pack).
	\end{itemize}
\section{Ejercicios Propuestos}
	\begin{itemize}		
		\item Forme grupos de 3 a 5 personas
		\item Un integrante del grupo deberá crear el proyecto principal, con el nombre de su grupo, con la plantilla base
		\item Comparta el proyecto con sus compañeros de grupo y asigne uno o dos  métodos distintos a cada integrante del grupo.
		\item Los integrantes del grupo deberán hacer clone, push y pull según corresponda, de modo que el repositorio contenga la solución final.
		\item Reportar al profesor que logró culminar la tarea. La tarea debe ser compartida con el profesor (CarloCorralesD) y entregada usando el mismo url que se usó para clonar el repositorio.
	\end{itemize}

	\section{Equipos, materiales y temas utilizados}
	\begin{itemize}
		\item Sistema operativo de 64 bits, procesador basado en x64.
		\item Visual Studio Code.
		\item Latex. 
		\item git version 2.41.0.windows.1
		\item Cuenta en GitHub con el correo institucional.
		\item Arreglos Estándar.	
	\end{itemize}
	\section{URL Github, Video}
	\begin{itemize}
		\item URL del Repositorio GitHub para clonar o recuperar.
		\item \url{https://github.com/FernandoGarambelM/Calculadora-pw2-.git}
		\item URL para el video flipgrid.
		\item \url{Url de flipgrid}	
	\end{itemize}
	\section{Actividades con el repositorio GitHub}
	
	\subsection{Creando e inicializando repositorio GitHub}
	\begin{itemize}	
		\item Como es el primer laboratorio se creo el repositorio GitHub.
		\item Se realizaron los siguientes comandos en la computadora:
	\end{itemize}	
		
	\begin{lstlisting}[language=bash,caption={Creando directorio de trabajo}][H]
		1. mkdir lluquecon
	\end{lstlisting}
	\begin{lstlisting}[language=bash,caption={Dirijíéndonos al directorio de trabajo}][H]
		1. cd lluquecon
	\end{lstlisting}	
	\begin{lstlisting}[language=bash,caption={Creando directorio para repositorio GitHub}][H]
		1. mkdir fp2-23b
	\end{lstlisting}
	\begin{lstlisting}[language=bash,caption={Inicializando directorio para repositorio GitHub}][H]
		1. pwd

		2. mkdir lluquecon

		3. CONTROL + L

		4. cd lluquecon

		5. mkdir fp2-23b

		6. cd fp2-23b

		7. git init

		8. ls -la
		
		9. git config --global user.name "Luis Guillermo Luque Condori"

		10. git config --global user.email lluquecon@unsa.edu.pe

		11. git config --list

		12. mkdir fase01

		13. mkdir fase02

		14. mkdir fase03

		15. cd fase01

		16. mkdir lab01

		17. cd lab01

		18. vim VideoJuego.java

		19. cd ..

		20. vim .gitignore

	\end{lstlisting}
	
	\subsection{Commits}
	\begin{itemize}	
		\item Se creo el archivo \textbf{.gitignore} para no considerar los archivos \textbf{*.class} que son innecesarios hacer seguimiento.
	\end{itemize}
	\begin{lstlisting}[language=bash,caption={Creando .gitignore}][H]
		$ vim lab01/.gitignore
	\end{lstlisting}
	\begin{lstlisting}[language=bash,caption={lab01/.gitignore}][H]
		*.class
	\end{lstlisting}
	\begin{lstlisting}[language=bash,caption={Commit: Creando .gitignore para archivos *.class}][H]
		$ git add .
		$ git commit -m "Creando .gitignore para archivos *.class"			
		$ git push -u origin main
	\end{lstlisting}
\section{1era sección}
\section{2da sección}
\section{3era sección}
	
	
	
	
	
	\section{\textcolor{red}{Rúbricas}}
	
	\subsection{\textcolor{red}{Entregable Informe}}
	\begin{table}[H]
		\caption{Tipo de Informe}
		\setlength{\tabcolsep}{0.5em} % for the horizontal padding
		{\renewcommand{\arraystretch}{1.5}% for the vertical padding
		\begin{tabular}{|p{3cm}|p{12cm}|}
			\hline
			\multicolumn{2}{|c|}{\textbf{\textcolor{red}{Informe}}}  \\
			\hline 
			\textbf{\textcolor{red}{Latex}} & \textcolor{blue}{El informe está en formato PDF desde Latex,  con un formato limpio (buena presentación) y facil de leer.}   \\ 
			\hline 
			
			
		\end{tabular}
	}
	\end{table}
	
	\clearpage
	
	\subsection{\textcolor{red}{Rúbrica para el contenido del Informe y demostración}}
	\begin{itemize}			
		\item El alumno debe marcar o dejar en blanco en celdas de la columna \textbf{Checklist} si cumplio con el ítem correspondiente.
		\item Si un alumno supera la fecha de entrega,  su calificación será sobre la nota mínima aprobada, siempre y cuando cumpla con todos lo items.
		\item El alumno debe autocalificarse en la columna \textbf{Estudiante} de acuerdo a la siguiente tabla:
	
		\begin{table}[ht]
			\caption{Niveles de desempeño}
			\begin{center}
			\begin{tabular}{ccccc}
    			\hline
    			 & \multicolumn{4}{c}{Nivel}\\
    			\cline{1-5}
    			\textbf{Puntos} & Insatisfactorio 25\%& En Proceso 50\% & Satisfactorio 75\% & Sobresaliente 100\%\\
    			\textbf{2.0}&0.5&1.0&1.5&2.0\\
    			\textbf{4.0}&1.0&2.0&3.0&4.0\\
    		\hline
			\end{tabular}
		\end{center}
	\end{table}	
	
	\end{itemize}
	
	\begin{table}[H]
		\caption{Rúbrica para contenido del Informe y demostración}
		\setlength{\tabcolsep}{0.5em} % for the horizontal padding
		{\renewcommand{\arraystretch}{1.5}% for the vertical padding
		%\begin{center}
		\begin{tabular}{|p{2.7cm}|p{7cm}|x{1.3cm}|p{1.2cm}|p{1.5cm}|p{1.1cm}|}
			\hline
    		\multicolumn{2}{|c|}{Contenido y demostración} & Puntos & Checklist & Estudiante & Profesor\\
			\hline
			\textbf{1. GitHub} & Hay enlace URL activo del directorio para el  laboratorio hacia su repositorio GitHub con código fuente terminado y fácil de revisar. &2 &X &2 & \\ 
			\hline
			\textbf{2. Commits} &  Hay capturas de pantalla de los commits más importantes con sus explicaciones detalladas. (El profesor puede preguntar para refrendar calificación). &4 &X &2 & \\ 
			\hline 
			\textbf{3. Código fuente} &  Hay porciones de código fuente importantes con numeración y explicaciones detalladas de sus funciones. &2 &X &1 & \\ 
			\hline 
			\textbf{4. Ejecución} & Se incluyen ejecuciones/pruebas del código fuente  explicadas gradualmente. &2 &X &2 & \\ 
			\hline			
			\textbf{5. Pregunta} & Se responde con completitud a la pregunta formulada en la tarea.  (El profesor puede preguntar para refrendar calificación).  &2 &X &2 & \\ 
			\hline	
			\textbf{6. Fechas} & Las fechas de modificación del código fuente estan dentro de los plazos de fecha de entrega establecidos. &2 &X &2 & \\ 
			\hline 
			\textbf{7. Ortografía} & El documento no muestra errores ortográficos. &2 &X &1 & \\ 
			\hline 
			\textbf{8. Madurez} & El Informe muestra de manera general una evolución de la madurez del código fuente,  explicaciones puntuales pero precisas y un acabado impecable.   (El profesor puede preguntar para refrendar calificación).  &4 &X &1 & \\ 
			\hline
			\multicolumn{2}{|c|}{\textbf{Total}} &20 & &13 & \\ 
			\hline
		\end{tabular}
		%\end{center}
		%\label{tab:multicol}
		}
	\end{table}
	
\clearpage

\section{Referencias}
\begin{itemize}			
	\item \url{https://www.w3schools.com/java/default.asp}
	\item \url{https://www.geeksforgeeks.org/insertion-sort/}
\end{itemize}	
	
%\clearpage
%\bibliographystyle{apalike}
%\bibliographystyle{IEEEtranN}
%\bibliography{bibliography}
			
\end{document}